%
% ================================================================
% 
\begin{frame}{\textbf{LAP} $\mid$ Definition $\mid$ The linear assignment problem}

    
    \begin{center}
$
\quad
\left [
\begin{array}{lcllllll}

\ &  \min z^k & =  & \displaystyle{\sum_{i=1}^{n}\sum_{j=1}^{n} c^k_{ij} x_{ij}}  & & k=1,\dots,p\vspace{2mm} \\

 & s/c  & &   \displaystyle{\sum_{i=1}^{n}  x_{ij}}   =  1 & \ &  j=1,\dots ,n \  \ \  \\
  &       & &   \displaystyle{\sum_{j=1}^{n}  x_{ij}}   =  1 & & i=1,\dots ,n \    \vspace{2mm} \\

 & & &  x_{ij} = (0,1) &&   i=1,\dots ,n  \vspace{-1mm}\\
 & & &   & & j=1,\dots ,n\\ 
 
\end{array}
\right ]
$ \hfill (p-LAP)
\end{center}
\end{frame}

%
% ================================================================
% 
\begin{frame}{LAP $\mid$ Definition $\mid$ Inputs}

Valid for 2-LAP.
\bigskip

              \begin{itemize}
                \item \texttt{n} (integer): \\ number $n$ of assignments task-resource 
                           \medskip
                %\item \texttt{p} (integer): \\ number $p$ of objectives, $k=1\dots p$                  
                \item \texttt{C1} (matrix of $n \times n$ of integers): \\  coefficients $c^1_{ij}$ of the objective 1  
                           \medskip
                \item \texttt{C2} (matrix of $n \times n$ of integers): \\  coefficients $c^2_{ij}$ of the objective 2                                         
              \end{itemize}
\end{frame}

%
% ================================================================
% 
\begin{frame}[fragile=singleslide]{LAP $\mid$ Data $\mid$ Example (console)}

{\small
\begin{verbatim}
n  =  5

C1 = [  3   9   0   0   6 ;
       16   0   6  12  19 ;
        2   7  11  15   8 ;
        4  11   7  16   3 ;
        2   5   1   9   0   ]

C2 = [ 16   5   6  19  12 ;
       15   7  13   7   7 ;
        1   2  13   2   3 ;
       14   7   8   1   7 ;
       10  10   1   0   0   ]
\end{verbatim}
}

\end{frame}

%
% ================================================================
% 
\begin{frame}[fragile=singleslide]{LAP $\mid$ Data $\mid$ Example (text file)}

\begin{columns}
\begin{column}{0.3\textwidth}
{\small
\begin{verbatim}
5
 3   9   0   0   6 
16   0   6  12  19 
 2   7  11  15   8 
 4  11   7  16   3 
 2   5   1   9   0   
16   5   6  19  12 
15   7  13   7   7 
 1   2  13   2   3 
14   7   8   1   7 
10  10   1   0   0 
\end{verbatim}
}
\end{column}
\begin{column}{0.3\textwidth}
\vspace{-2mm}
%\begin{center}
\hspace{-5mm}
\begin{tikzpicture}[scale=0.625]%[auto]

\draw  [<->, color=blue](0,-1) -- (0,1.85);
\draw  [<->, color=blue](0,2.10) -- (0,5.0);
\draw  [<->, color=blue](0,5.2) -- (0,5.7);
\node [text width=1cm,align=left, color=blue] at (1.5,5.45) {{\texttt{n}}};
\node [text width=1cm,align=left, color=blue] at (1.5,3.6) {{\texttt{C1}}};
\node [text width=1cm,align=left, color=blue] at (1.5,0.5) {{\texttt{C2}}};
\draw  [-, dotted, color=blue](-1,2) -- (1,2);
\draw  [-, dotted, color=blue](-1,5.1) -- (1,5.1);
\end{tikzpicture}
\vfill
\end{column}
\begin{column}{0.15\textwidth}

\end{column}
\end{columns}

\vspace{10mm}
\begin{itemize}
%\item  \blue{createLAP}\\
%          Create a new instance of a LAP     \\
%          \texttt{id = createLAP(  ) }
\item  \blue{load2LAP}\\
          Load an instance of a 2-LAP from the file \texttt{fname}    \\
           \texttt{id = \texttt{{load2}LAP( fname ) }} 
           \medskip
\end{itemize}

\end{frame}

%
% ================================================================
% 
\begin{frame}{LAP $\mid$ Resolution $\mid$ API}

\begin{itemize}
%\item  \blue{createLAP}\\
%          Create a new instance of a LAP     \\
%          \texttt{id = createLAP(  ) }
\item  \blue{set2LAP}\\
          Create a new instance of a 2-LAP and set up all required values    \\
           \texttt{id = \texttt{{set2}LAP( n , C1, C2 ) }} 
           \medskip
%\item  \blue{set2LAP}\\
%          Set up all required values of a 2-LAP\\
%          \texttt{set2LAP( id , n , C1, C2 ) } 
\item  \blue{LAP\_Przybylski2008} \\
          Set up the solver to use for the 2-LAP\\
          \texttt{solver = LAP\_Przybylski2008( ) }
          \medskip
\item  \blue{solveMOP}\\
          Solve the instance provided with the mentioned solver and return the results \\
          \texttt{z1, z2, $\sigma$ = solveMOP( id , solver ) }
%          \medskip
%\item  \blue{get\red{2}LAP}\\
%          Retrieve   \\
%          \texttt{results = get\red{2}LAP( id ) }         
\end{itemize}

\end{frame}

%
% ================================================================
% 
\begin{frame}{LAP $\mid$ Resolution $\mid$ Outputs (specification)}

Valid for 2-LAP.
\bigskip

solveMOP returns:
\medskip
                           \begin{itemize}
                            %  \item $CPUt$: \\ the time consumed
%                              \item $nYn$: \\ the number of non-dominated points
                              \item z1: vector of $1\dots \mid Y_N \mid $ of integers
                              \item z2: vector of $1\dots \mid Y_N \mid $ of integers
                              \item $\sigma$: vector of $1,\dots, \mid Y_N \mid $ of  ($\sigma_1, \dots, \sigma_n$)
                                                                                         
%                              $$ z_1, z_2, \sigma_1, \dots, \sigma_n$$
                              where
                               \begin{itemize}
%                                     \item $z_k, \ k=1,\dots,p$ of integers:  performances 
                                     \item $\sigma_i (\ i=1,\dots,n$): integer;  a permutation coding ($x_{ij}=1 \Leftrightarrow \sigma_i=j$)
                                \end{itemize}
                           \end{itemize}               

\end{frame}

%
% ================================================================
% 
%\begin{frame}{LAP $\mid$ Resolution $\mid$ Outputs (text file)}
%\red{a ecrire}
%\end{frame}
